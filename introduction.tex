\section{Introduction}

The study and prediction of environmental and geophysical flow
processes requires knowledge of the spatially and temporally varying
flow field.  Large scale, predominantly horizontal flow fields in such
application fields are usually computed through the numerical solution
of the depth-integrated shallow water equations (SWE).  For example,
SWE-based models are used for tsunami and coastal inundation
predictions in \cite{Marras:2016, Vater2019, Qin:2019}, flood
predictions \cite{George2011, Echeverribar2019}, and river hydraulics
\cite{Persi:2019}.  More recently, the SWE have been used in
rainfall-runoff studies in small natural catchments \cite{Mugler2011,
  Lacasta2014, Simons2014, Xia2019, CaviedesVoullieme2020}.

Many of the aforementioned SWE applications involve wetting and drying
processes.  For example, coastal simulations involve wave runup and
overtopping \cite{Vater20151, Medeiros2013, Vater2019}. In
rainfall-runoff simulations, cells frequently dry out and get flooded
\cite{Simons2014, Lacasta2014, Xia2017}. Flood simulations feature a
flood front, which switches cells from dry to wet as it propagates
through the domain \cite{George2011}.  The interface between a wet and
a dry cell is usually referred to as a wet/dry interface
\cite{Bollermann2013, Beisiegel2015} and requires special
considerations. TODO: literature review

Another numerical issue arises in realistic SWE applications due to
rough topography.  In the mathematical model of the SWE, the
topography is accounted for through a bed slope source term
\cite{Kesserwani2013}.  If fluxes and source terms are not carefully
balanced, the numerical scheme is unable to maintain equilibrium
states.  The most relevant equilibrium condition is the so-called
``lake-at-rest'' equilibrium \cite{bermude2}.  Many numerical
treatments that address this issue have been proposed in the
literature, especially in the finite volume framework.  For example,
the hydrostatic reconstruction \cite{audusse}, the surface gradient
method \cite{zhou}, the f-wave solvers \cite{George:2008,
  LeVeque:2011}, the path-conservative approaches \cite{Gosse:2001,
  LeVeque:2011}, and the augmented solvers \cite{NavasMontilla2020},
among many others.

In this work, we develop a robust, third-order accurate discontinuous
Galerkin (DG) method with a novel stability criterion. In recent
years, DG methods have gained significant interest as a means to
numerically solve the SWE \cite{CaviedesVoullieme2015, kesserwani2015,
  Vater20151, Marras:2016, Kesserwani2019, Vater2019,
  CaviedesVoullieme2020, NavasMontilla2020}.  The DG method uses a
local piecewise-polynomial representation of the solution, which is
driven by fluxes across cell interfaces.  Thus, the DG method is
locally conservative.  Higher order of accuracy is achieved by
increasing the polynomial order.  Such high order schemes require
additional numerical treatment to prevent them from producing
non-physical oscillations.

A well-known approach in the theory of higher order finite volume and
finite difference methods is the use of slope limiters to ensure that
the resulting scheme is total variation diminishing (TVD)---that is to
say that the total variation of the solution decreases in a monotonic
way \cite{leveque2}.  As globally applying slope limiters to the
solution leads to loss of accuracy, limiters are usually applied only
in so-called ``troubled cells'' \cite{krivodonova2004, shu2005}.
However, the detection of troubled cells is not trivial.  In
\cite{Ambati2007452, krivodonova2004}, troubled cells are detected
based on the magnitude of the discontinuity. The criterion is enhanced
in \cite{kesserwani2015} by using a monotonicity condition.  The
disadvantage of this discontinuity criterion is that the discontinuity
in the solution is compared with a predefined constant to detect the
troubled cells. However, the value of this constant seems to depend on
the study case.  Other approaches to ensure stability are, for
example, limiting-dissipation operators \cite{JAFFRE1995,
  Ambati2007452} and artificial viscosity \cite{cesenek2013,
  bublik2011, Bublik2015329}.

In this work, we propose a novel criterion to detect troubled cells.
This criterion is based on the shape of the solution in the cell and
does not require any user-specified constant.  In the detected
troubled cells, we apply the strategy from \cite{Cockburn1989b}, where
the solution's accuracy is decreased to second order and then further
limited using a minmod-limiter.  Equilibria are preserved through the
surface gradient method \cite{zhou}.  Wet/dry interfaces are treated
following the positivity preserving reconstruction in \cite{kurg2}.

The rest of the paper is structured as follows.  Firstly, we present
the governing equations and the DG method.  This is followed by a
presentation of the novel criterion to detect troubled cells and
discussions of the surface gradient method and the wet/dry interface
treatment.  The resulting scheme is tested in a set of benchmarks.
Finally, conclusions are given.
