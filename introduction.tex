\section{Introduction}

The study and prediction of environmental and geophysical flow
processes requires knowledge of the spatially and temporally varying
flow field.  Large scale, predominantly horizontal flow fields in such
application fields are usually computed through the numerical solution
of the depth-integrated shallow water equations (SWE).  For example,
SWE-based models are used for tsunami and coastal inundation
predictions in \cite{Marras:2016, Vater2019, Qin:2019}, flood
predictions \cite{George2011, Echeverribar2019}, and river hydraulics
\cite{Persi:2019}.  More recently, the SWE have been used in
rainfall-runoff studies in small natural catchments \cite{Mugler2011,
  Lacasta2014, Simons2014, Xia2019, CaviedesVoullieme2020}.

Many of the aforementioned SWE applications involve wetting and drying
processes.  For example, coastal simulations involve wave runup and
overtopping \cite{Vater20151, Medeiros2013, Vater2019}. In
rainfall-runoff simulations, cells frequently dry out and get flooded
\cite{Simons2014, Lacasta2014, Xia2017}. Flood simulations feature a
flood front, which switches cells from dry to wet as it propagates
through the domain \cite{George2011}.  The interface between a wet and
a dry cell is usually referred to as a wet/dry interface
\cite{Bollermann2013, Beisiegel2015} and requires special
considerations. TODO: 

The mathematical model of shallow water equations contains the source
terms. The bed slope source term is caused by the bed
irregularities. This source term is well described in many works like
\cite{Ambati2007452,Ambati20071233,kesserwani2015,Beisiegel2015,Tassi2007998}.

The proper numerical treatment of these wet/dry interfaces is
directly related to the crucial properties of (i) mass conservation,
(ii) C-property, (iii) well-balancedness, (iv) positivity-preserving,
and (v) shock-capturing.  In order to ensure an accurate solution, the
numerical scheme must satisfy these properties
\cite{CaviedesVoullieme2020}.

In recent years, discontinuous Galerkin (DG) methods have gained
significant interest as a means to numerically solve the SWE
\cite{CaviedesVoullieme2015, kesserwani2015, Vater20151, Marras:2016,
  Kesserwani2019, Vater2019, CaviedesVoullieme2020,
  NavasMontilla2020}.  The DG method uses a local piecewise-polynomial
representation of the solution, which is driven by fluxes across cell
interfaces.  Thus, the DG method is locally conservative.  Higher
order of accuracy is achieved by increasing the polynomial order.


In most of these cases wetting and drying processes of the cells occur
during the simulation, e.g. part of the computational domain may fall
dry or the domain may start initially dry and get flooded.   In
\cite{Vater20151} the wet/dry interface treatment is described for the
second order DGFEM. In this work the treatment suitable for the third
order DGFEM is suggested.


Except for wet/dry interface and bed slope source term, special
treatment is also needed if the shocks and discontinuities are present
in the computational domain. High order accuracy schemes have to be
limited to prevent them from the appearance of non-physical
oscillations. In the theory of the finite volume method with linear
reconstruction of the conservative variables, some TVD-minmod limiter
restricts the variation of the local slope of the linear approximation
with respect to the upstream and downstream gradients. This is
referred to as a global limiting process. Accordingly to
\cite{krivodonova2004,shu2005} this global limiting approach leads to
the loss of accuracy. Thus the limiter is used only in 'troubled'
cells. In \cite{krivodonova2004} the discontinuity detection scheme
was used to define these cells. The same criterion was used in
\cite{Ambati2007452}. In \cite{kesserwani2015}, Kesserwani added the
criterion of monotonicity of the solution. Disadvantage of the
discontinuity criterion is the definition of the discontinuity
itself. Discontinuity in the solution is compared with a predefined
constant which defines the 'troubled' cells. Unfortunately this
constant can be arbitrarily large and depends on the study case.
Within this work we suggest novel criterion of defining 'troubled'
cells. This criterion is based on the shape of the solution in the
finite element cell and not depends on any constant.

In the 'troubled' cells the limiting process is implemented. In
\cite{JAFFRE1995,Ambati2007452}, the limiting-dissipation operator was
used to limit the numerical solution. Artificial viscosity was used
for example in \cite{cesenek2013, bublik2011,Bublik2015329}. Within
this work, the process described in \cite{Cockburn1989b} is used. The
solution is limited by decreasing of the order of accuracy to the
second order. This second order solution was then limited by minmod
limiter. Finite volume limiters were also adopted in \cite{yang2009}.

The novel criterion of defining 'troubled' cells and wet/dry interface
treatment are tested and validated by the classical benchmarks and
compared by other numerical methods and analytical solutions at the
end of this paper.
